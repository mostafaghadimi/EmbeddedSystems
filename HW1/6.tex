\سؤال{}


\begin{itemize}
	\item الف)
	\begin{itemize}
		\item \lr{Hard Real-time}: به سیستم‌هایی گفته می‌شود که ضمانت شود به همه‌ی ددلاین‌ها در محدوده زمانی مشخص شده، پاسخ می‌دهد؛ مانند سیستم‌های هسته‌ای، سیستم‌های برقی هواپیما، دستگاه‌های ضربان‌ساز در پزسکی و ...
		\item \lr{Firm Real-time}:
		به سیستم‌هایی گفته می‌شود که تعداد از دست دادن ددلاین‌ در آن‌ها بسیار کم است و به عنوان سیستمی که \lr{fail} شده آن‌ها را در نظر نمی‌گیرند.  در این مواقع به دلیل فاصله مناسب خطاها، سیستم می‌تواند نجات پیدا کند، در حالی‌که ارزش انجام کار به صفر می‌رسد یا غیرممکن می‌شود؛ مانند تولید سیستم‌های دارای خط مونتاژ روباتی که از دست دادن ددلاین باعث درست وصل نشدن قطعه می‌شود. اما تا زمانی که تعداد قطعات خراب به اندازه کافی کم و نادر باشند که  توسط کنترل کیفیت رد بشوند و هزینه زیادی هم نداشته باشند، تولید ادامه دارد. یک مثال دیگر می‌تواند کابل‌های دیجیتال \lr{set-up box} باشد که زمان را موقعی که نیاز است تا روی صفحه فریم‌ها نمایش داده شوند، رمزگشایی می‌کند. چون فریم‌ها به ترتیب زمانی حساسند، از دست دادن یک ددلاین باعث لرزش و کاهش کیفیت تصویر می‌شود. اگر فریم از دست رفته بعدا در دسترس قرار بگیرد، نمایش آن باعث لرزش بیش‌تر تصویر می‌شود بنابراین بی‌فایده است. بیننده تا زمانی که تعداد این لرزش‌ها زیاد و مکرر نباشد، از برنامه می‌تواند لذت ببرد.
		\item \lr{Soft Real-time}:
		به سیستم‌هایی گفته می‌شود که مکررا در آن‌ها ممکن است ددلاین‌ها از دست می‌روند اما چون تسک‌ها به‌طور زمان‌بندی‌شده اجرا می‌وند نتایج آن‌ها مقادیری دارد. 
		مثلا در ایستگاه‌های هواشناسی سنسورهای زیادی برای خواندن و اندازه‌گیری دما، رطوبت، سرعت باد و ... وجود دارد. خواندن اطلاعات آن‌ها باید در بازه‌های زمانی مشخص انجام گیرد و انتقال داده شوند. در صورتی که این فرآیند، هم‌آهنگ نیست. با این وجود، هم‌چنان این عددها می‌توانند به اندازه‌ی کافی نزدیک باشند. نمونه دیگر  آندر کنسول‌های بازی است که یک نرم‌افزار را برای موتور بازی اجرا می‌کند. منابع زیادی وجود دارند که باید بین تسک‌ها به اشتراک گذاشته شوند. هم‌چنین، این تسک‌ها باید کامل شوند با توجه به برنامه‌ی بازی تا به‌درستی نمایش داده شوند. در صورتی که بتوانند نزدیک به هم اجرا شوند که لذت‌بخش خواهد بود اما در غیر این‌صورت در بازی وقفه‌ی کوچکی به وجود می‌‌آید. 
	\end{itemize}

	\item ب) 
	سیستم‌عامل‌های همه‌منظوره یک جزء ضروری در هر دستگاه موبایل، سیستم‌کامپیوتری و... است. برای انجام چندین کار به‌طور هم‌زمان بسیار عالی عمل می‌کنند ولی مشکلاتی نظیر تاخیر و هم‌آهنگی آن‌ها را برای کاربردهایی که به زمان حساس هستند، غیر ایده‌آل می‌کند.
	
	در مقابل سیستم‌عامل‌های بی‌درنگ پلتفرم‌های نرم‌افزاری هستند که برای کاربردهایی طراحی شده‌اند که زمان در آن‌ها اهمیت زیادی دارد. 
\begin{itemize}
	\item \textbf{\lr{RTOS}}
	\begin{itemize}
		\item برنامه‌ریزی همیشه بر اساس اولویت است.
		\item یک تسک با اهمیت کم‌تر توسط یک تسک با اهمیت بیش‌تر متوقف می‌شود؛ حتی اگر در حال اجرا کردن \lr{kernel} باشد.
		\item در جایی که توسعه مهم است، کد \lr{kernel} سیستم‌عامل‌های بی‌درنگ طوری طراحی شده‌اند تا مقیاس‌پذیر باشند، بنابراین توسعه‌دهنده می‌تواند این کار را به راحتی انجام دهد.
		\item اجرای برنامه‌ها قطعی \footnote{\lr{deterministic}} است و هیچ الگوی اجرای تصادفی در آن وجود ندارد.
		\item زمان اجرای و دریافت پاسخ قابل پیش‌بینی است.
		\item محدودیت زمانی دارد.
		\item چون از آن‌ها در میکروکنترلرها استفاده می‌شود، نباید بیش‌تر از ده درصد از فضای حاقظه را اشغال کند.
		\item به‌طور بهینه پیاده‌سازی نمی‌شود.
	\end{itemize} 

	\item \textbf{\lr{GPOS}}
	\begin{itemize}
		\item برنامه‌ریزی همیشه بر اساس اولویت  برنامه‌ریزی تسک‌ها همیشه براساس اولویت و اهمیت نیست.
		\item هر چه تعداد ترد بیش‌تر باشد، برنامه‌ریزی و شروع اجرای آن‌ها زمان بیش‌تری می‌گیرد.
		\item در جایی که توسعه مهم است، در حالت کلی کد سیستم‌عامل‌های همه‌منظوره ماهیت \lr{modular} ندارد.
		\item الگوی اجرایی تصادفی دارد.
		\item تضمینی برای مدت زمان دریافت پاسخ وجود ندارد.
		\item نگاشت حافظه به صورت پویا انجام می‌شود. \footnote{\lr{Dynamic Memory Mapping}}
		\item به‌طور میانگین برای اکثر کارها بهینه پیاده‌سازی شده است. \footnote{\lr{average case}}
		\item حاقظه زیادی دارد.
	\end{itemize}
	با توجه به مقایسه‌ای که انجام شده، بسته به کاربرد هر کدام از این‌ها می‌توانند مزیت و یا عیب در نظر گرفته شوند.  برای نمونه سیستم‌عامل‌های \lr{RTOS} چون کم‌ترین تعداد تسک‌ها رو انجام می‌دهند برای کاربردهایی که انجام چندین تسک به‌طور هم‌زمانن، پردازش سبک اما با سرعت و کارایی مد نظر است، به کار نمی‌آیند.
\end{itemize}
	
	\item ج)
	\begin{itemize}
		\item سیستم کنترل آلودگی هوا
		\item سیستم رزرو هواپیمایی
		\item صنایع دفاعی مانند رادار
		\item سیستم‌هایی که فورا به‌روزرسانی می‌شوند.
		\item و ...
	\end{itemize}
\end{itemize}