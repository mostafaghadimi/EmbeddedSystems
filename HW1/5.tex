\سؤال{}

میکروکنترلر \lr{AVR} دارای معماری \lr{RISC} \footnote{\lr{Reduced Instruction Set Architecture}} می‌باشد. در هر کلاک می‌توانند یک دستور ساده انجام دهند؛ مانند انتقال عدد از حافظه به رجیستر. این میکروکنترلرها معمولا ۸ بیتی هستند. از آن‌ها برای انجام کارهای کنترلی استفاده می‌شود. به بیان دیگر در جاهایی که نیاز به محاسبات پردازشی وجود ندارد، از آن‌ها استفاده می‌شود؛
مانند کنترل روشنایی

میکروکنترلر \lr{ARM} دارای معماری \lr{RISC} می‌باشد. این میکروکنترلرها معمولا ۳۲ بیتی هستند. به دلیل توانایی بالا در اجرای سیستم‌عامل‌های مختلف مانند لینوکس، دارای کاربرد وسیعی در حوزه‌های مختلف می‌باشند؛ مانند:
\begin{itemize}
	\item تبلت
	\item دوربین‌های دیجیتال
	\item تلفن‌های همراه
	\item و...
\end{itemize}

میکروکنترلر ۸۰۵۱، ۸ بیتی است. دارای فضای \lr{stack} بسیار محدود ۱۲۸ بایتی است که حتی نوشتن یک کامپایلر برای زبان \lr{C} را در آن‌ها با چالش روبه‌رو می‌کند. به ازای هر \lr{instruction} به چندین \lr{clock} نیاز دارد و توسط شرکت \lr{Intel} طراحی شده است.

\begin{itemize}
	\item \textbf{\lr{Bus width}}
	\begin{itemize}
		\item \lr{8051}:  ۸ بیتی
		\item \lr{AVR}: ۸ / ۳۲ بیتی
		\item \lr{ARM}: اغلب ۳۲ بیتی است اما ۶۴ بیتی هم دارد.
	\end{itemize}

	\item \textbf{سرعت}
	\begin{itemize}
		\item \lr{8051}: 
		$12 \: clock / instruction \: cycle$
		\item \lr{AVR}:
		$1\:clock / instruction\:cycle$
		\item \lr{ARM}:
		$1\:clock / instruction\:cycle$
	\end{itemize}

	\item \textbf{\lr{ISA}}
	\begin{itemize}
		\item \lr{8051}: \lr{CLSC}
		\item \lr{AVR}: \lr{RISC} 
		\item \lr{ARM}: \lr{RISC}
	\end{itemize}
	
	\item \textbf{مصرف انرژی}
	\begin{itemize}
		\item \lr{8051}: متوسط
		\item \lr{AVR}: کم
		\item \lr{ARM}: کم
	\end{itemize}

	\item \textbf{\lr{Memory Architecture}}
	\begin{itemize}
		\item \lr{8051}: \lr{Von Neumann}
		\item \lr{AVR}: \lr{Modified}
		\item \lr{ARM}: \lr{Modified Harvard Architecture}
	\end{itemize}
\end{itemize}