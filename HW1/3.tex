\سؤال{}

\begin{itemize}
	\item الف)
	به هزینه‌ی اولیه‌ای که صرف تحقیق، طراحی، ایجاد و تست یک محصول جدید یا بهبود محصول می‌شود، اصطلاحا هزینه‌ی \lr{NRE} یا \lr{Non-Recurring Engineering} گفته می‌شود. با وجود این‌که این هزینه فقط یک‌بار برای محصول صرف می‌شود، اما هنگام بررسی سوددهی حتما باید در نظر گرفته شود.
	\item ب)
	\begin{itemize}
		\item اگر بخواهیم از ریزپردازه استفاده کنیم:
		\begin{equation*}
		cost = 5000 + 25x
		\end{equation*}
		\item 
		اگر بخواهیم از \lr{ASIC} استفاده کنیم:
		\begin{equation*}
			$$
			cost = 100000 + 5x
			$$
		\end{equation*}
	\end{itemize}
	با مساوی قرار دادن این دو معادله به دست می‌آوریم که $x = 4750$ و به این معنی است که اگر بخواهیم تا ۴۷۵۰ محصول تولید کنیم، استفاده از ریزپردازه به‌صرفه‌تر است، در غیر این‌صورت (بیش از ۴۷۵۰ محصول)، استفاده از \lr{ASIC}.
\end{itemize}