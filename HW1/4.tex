\سؤال{}


یک پردازنده‌ی سیگنال دیجیتال، همان‌طور که از نامش مشخص است سیگنال‌های دیجیتال مانند صدا، ویدیو، دما، و... را دریافت می‌کند و بر روی آن‌ها عملیات ریاضی انجام می‌دهد. این دستگاه‌ها برای انجام عملیات سریع ریاضی (مانند جمع، تفریق، ضرب و تقسیم) بهینه‌سازی و ساخته شده‌اند.
 سگنال‌ها در آن پردازش می‌شوند بنابراین اطلاعاتی که دارند می‌توانند نمایش داده شوند، تحلیل و بررسی روی آن‌ها صورت گیرد و یا به یک سیگنال با نوع دیگر تبدیل شوند؛ برای مثال صدا یک سیگنال آنالوگ است و نیاز است تا ابتدا به کمک یک تبدیل‌کننده، به سیگنال دیجیتال تبدیل شده و به عنوان ورودی به \lr{Digital Signal Processor} داده شود.
 
از مهم‌ترین الگوریتم‌هایی که در پردازش سیگنال‌های دیجیتال استفاده می‌شود، می‌توان به موارد زیر اشاره کرد:
\begin{itemize}
	\item \lr{Fast Fourier Transform}
	\item \lr{Finite Impulse Response Filters}
\end{itemize}

برای کاربردهای پردازش سیگنال‌های دیجیتال نیز می‌توان به موارد زیر اشاره کرد:
\begin{itemize}
	\item پردازش گفتار
	\item استفاده در رادار
	\item ارتباطات راه دور
	\item پردازش صدا
	\item پردازش تصویر
	\item ردیاب آوایی
	\item و...
\end{itemize}